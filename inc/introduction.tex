Consider a time-dependent partial differential equation (PDE)
\begin{equation}
\frac{\partial{u}}{\partial{t}} + Au = f\quad \textnormal{in the time interval}\ [0, T],
\end{equation}
with initial conditions $u(t=0) = u_0$ and for now with some unspecified boundary conditions.
The unknown $u$ can be scalar or vectorial and the PDE may be of linear or non-linear type.
We propose in this note a time discretization scheme which intrinsically parallelizes and
meets a desired a-priori prescribed precision. To this end, choose an integral number 
$N \in \mathbb{N}$ representing the number of time steps with length $\Delta T = T / N$ 
and write $T^n = n \Delta T$ for $n=0, ... ,N$. Next the \textit{functions} $\lambda_n$ for
$n=0, ..., N-1$ are introduced for each time interval (certainly $\lambda_0 = u_0$) and one
solves, within the interval $[T^n, T^{n+1}]$ the equation
\begin{equation}
\frac{\partial u_n}{\partial t} + A u_n = f_n, \quad f_n = f_{|[T^n, T^{n+1}]},
\end{equation}
with the initial conditions $u_n(t=T^n)=\lambda_n$ with the same boundary conditions as
previously for $u$. Thus the collection $u_n$ can be solved in parallel for $n=0, ..., N-1$.
This collection coincides with 
$\{u_{|[T^n, T^{n+1}]}\}_n$ when $u_n(T^{n+1,-})\ (=\lim_{\delta>0, \delta \to 0} u_n(T^{n+1} - \delta)) = \lambda_{n+1}$
for $n=0, ..., N-1$. We show in the following how to find the values in an iterative way.

\paragraph{Remark} One could consider that the $\lambda_n$ represent a `virtual control'. 
Also is it natural to introduce the function of costs for the `virtual control' by 
$J(\lambda) = \sum_{n=1}^{N-1} ||u_n(T^n,-) - \lambda_n||^2$ and using a method of gradient
for finding the $\lambda_n$ by minmization the functional $J(\lambda)$. However one major
drawback of this formulation is that it is no canonically way to parallelize the method.
Therefore we propose a scheme which incorporates the feedback of the cost function for
finding the values yet allowing to parallelize the scheme more easily.

For illustration purposes consider a simple linear differential equation 
\begin{equation}
\begin{cases}
\frac{dy}{dt}(t) = -ay(t) \quad \textnormal{in}\ [0, T], \\
y(t=0) = y_0.
\end{cases}
\end{equation}
Make use of the implicit Euler method
\begin{equation}
\frac{Y^{n+1} - Y^{n}}{\Delta T} + aY^{n+1} = 0, \\
Y^0 = y_0,
\end{equation}
then utilize the precalculated (coarse) values for 
solving exact (or very fine) on every time interval $[T^n, T^{n+1}]$ 
\begin{equation}
\begin{cases}
\frac{dy^n}{dt}(t) = -ay^n(t)\quad \textnormal{in}\ [T^n, T^{n+1}], \\
y^n(t=T^n) = Y^n.
\end{cases}
\end{equation}
Next an iterative method is proposed to increase 
the precision for the preliminary scheme.

Let $Y_1^n = Y^n$ and define on $[T^n, T^{n+1}]$ $y_1^n(t) = y^n(t)$. Next
suppose $Y_k^n$ and $y_k^n(t)$ are known on $]T^n, T^{n+1}]$,
\begin{enumerate}
\item[(i)] Introduce the jumps $S_k^n = y_k^{n-1}(T^n)-Y_k^n$,
\item[(ii)] propagate the jumps
\begin{equation}
\label{eq:propagation_of_jumps}
\begin{cases}
\frac{\gamma_k^{n+1} - \gamma_k^n}{\Delta T} + a \gamma_k^{n+1} = \frac{S_k^n}{\Delta T},\\
\gamma_k^0 = 0,
\end{cases}
\end{equation}
\item[(iii)] finally set $Y_{k+1}^n = y_k^{n-1}(T^n) + \gamma_k^n$ and solve exact and in parallel
\begin{equation}
\begin{cases}
dy_{k+1}^n(t) = -ay_{k+1}^n(t) \quad in [T^n, T^{n+1}], \\
y_{k+1}^n(t = T^n) = Y_{k+1}^n.
\end{cases}
\end{equation}
\end{enumerate}
The previous demands for a proof of the following:
\paragraph{\textsc{PROPOSITION 1}} \textit{The preceeding scheme is of order $k$ given a constant $c_k$ such that:}
\begin{equation}
\forall n, 0 \leq n \leq N-1, \quad |Y_k^n-y(T^n)|+\max_{t\in[T^n,T^{n+1}]}|y_k^n(t) - y(t)| \leq c_k \Delta T^k.
\end{equation}

\paragraph{\textit{Proof}} The proposition is trivial for $rank(k) = 1$, we will show this by recursion. 
Remember that $y_k^{n-1}(T^n) = e^{-a\Delta T}Y_k^{n-1}$ and that $S_k^n = e^{-a\Delta T}Y_k^{n-1} - Y_k^n$ holds.
Furthermore the solution of \eqref{eq:propagation_of_jumps} is
\begin{equation}
\gamma_k^n = \sum_{p=1}^{n-1} (1+a\Delta T)^{p-n}S_k^p = \sum_{p=1}^{n-1} (1+a \Delta T)^{p-n}(e^{-a\Delta T} Y_k^{p-1} - Y_k^p),
\end{equation}
as well as 
\begin{equation}
Y_{k+1}^n = e^{-a\Delta T}Y_k^{n-1} + \sum_{p=1}^{n-1}(1+a\Delta T)^{p-n}(e^{-a\Delta T}Y_k^{p-1} - Y_k^p)
\end{equation}
holds. Given the exact solution $y(T^n) = e^{-na\Delta T}y_0$ 
one can deduce that ...
