\documentclass[12pt]{article}
\usepackage{listings}  
\usepackage{color}
\usepackage[numbered]{matlab-prettifier}
\usepackage{hyperref}
\usepackage{amssymb}
\usepackage{amsmath} % for cases
\usepackage{graphicx}
\usepackage{epstopdf}
\usepackage[printwatermark]{xwatermark}
\usepackage{float}
\hypersetup{
    colorlinks,
    citecolor=black,
    filecolor=black,
    linkcolor=black,
    urlcolor=black,
    pdftitle    = {English Language Proficiency Exam},
    pdfsubject  = {Translation of a French paper to English in Numerical Analysis},
    pdfauthor   = {Stephan Grein},
    pdfkeywords = {Translation, English, Frensh, Numerical Analysis, time-parallel methods, Parareal, PDE, ODE}
}

\title{Solution of a PDE by a `parareal' scheme \footnote{Translation of the original French research article \cite{Lions2001} to English by Stephan Grein}}
\author{Jacques-Louis LIONS, Yvon MADAY and Gabriel TURINIC}
\newwatermark*[allpages,color=blue!50,angle=45,scale=2,xpos=0,ypos=0]{DRAFT (Stephan Grein)}

\begin{document}
\maketitle

\begin{abstract}
The purpose of this Note is to propose a time discretization of a partial differential evolution
equation that allows for parallel implementations. The method, based on an Euler scheme,
combines coarse resolutions and independent fine resolutions in time in the same spirit
as standard spacial approximations. The resulting parallel implementation is done in the
non standard time direction. Its main goal concerns real time problems, hence the proposed
terminology of `parareal' algorithm.
\end{abstract}

\tableofcontents
\newpage

\section{Introduction}
Consider a time-dependent partial differential equation (PDE)
\begin{equation}
\frac{\partial{u}}{\partial{t}} + Au = f\quad \textnormal{in the time interval}\ [0, T],
\end{equation}
with initial conditions $u(t=0) = u_0$ and for now with some unspecified boundary conditions.
The unknown $u$ can be scalar or vectorial and the PDE may be of linear or non-linear type.
We propose in this note a time discretization scheme which intrinsically parallelizes and
meets a desired a-priori prescribed precision. To this end, choose an integral number 
$N \in \mathbb{N}$ representing the number of time steps with length $\Delta T = T / N$ 
and write $T^n = n \Delta T$ for $n=0, ... ,N$. Next the \textit{functions} $\lambda_n$ for
$n=0, ..., N-1$ are introduced for each time interval (certainly $\lambda_0 = u_0$) and one
solves, within the interval $[T^n, T^{n+1}]$ the equation
\begin{equation}
\frac{\partial u_n}{\partial t} + A u_n = f_n, \quad f_n = f_{|[T^n, T^{n+1}]},
\end{equation}
with the initial conditions $u_n(t=T^n)=\lambda_n$ with the same boundary conditions as
previously for $u$. Thus the collection $u_n$ can be solved in parallel for $n=0, ..., N-1$.
This collection coincides with 
$\{u_{|[T^n, T^{n+1}]}\}_n$ when $u_n(T^{n+1,-})\ (=\lim_{\delta>0, \delta \to 0} u_n(T^{n+1} - \delta)) = \lambda_{n+1}$
for $n=0, ..., N-1$. We show in the following how to find the values in an iterative way.

\paragraph{Remark}.- One could consider that the $\lambda_n$ represent a `virtual control'. 
Also is it natural to introduce the function of costs for the `virtual control' by 
$J(\lambda) = \sum_{n=1}^{N-1} ||u_n(T^n,-) - \lambda_n||^2$ and using a method of gradient
for finding the $\lambda_n$ by minmization the functional $J(\lambda)$. However one major
drawback of this formulation is that it is no canonically way to parallelize the method.
Therefore we propose a scheme which incorporates the feedback of the cost function for
finding the values yet allowing to parallelize the scheme more easily.

For illustration purposes consider a simple linear differential equation 
\begin{equation}
\begin{cases}
\frac{dy}{dt}(t) = -ay(t) \quad \textnormal{in}\ [0, T], \\
y(t=0) = y_0.
\end{cases}
\end{equation}
Make use of the implicit Euler method
\begin{equation}
\frac{Y^{n+1} - Y^{n}}{\Delta T} + aY^{n+1} = 0, \\
Y^0 = y_0,
\end{equation}
then utilize the precalculated (coarse) values for 
solving exact (or very fine) on every time interval $[T^n, T^{n+1}]$ 
\begin{equation}
\begin{cases}
\frac{dy^n}{dt}(t) = -ay^n(t)\quad \textnormal{in}\ [T^n, T^{n+1}], \\
y^n(t=T^n) = Y^n.
\end{cases}
\end{equation}
Next an iterative method is proposed to increase 
the precision for the preliminary scheme.


\section{Comments on stability}
\dots
\section{Application of the `parareal' scheme to linear PDEs}
\dots
\section{Analogue application to non-linear PDEs}
\dots

\bibliographystyle{plain}
\bibliography{bib}

%%% content here
\end{document}
