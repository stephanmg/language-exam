\documentclass[12pt]{article}
\usepackage{listings}  
\usepackage{color}
\usepackage[numbered]{matlab-prettifier}
\usepackage{hyperref}
\usepackage{amssymb}
\usepackage{graphicx}
\usepackage{epstopdf}
\usepackage[printwatermark]{xwatermark}
\usepackage{float}
\hypersetup{
    colorlinks,
    citecolor=black,
    filecolor=black,
    linkcolor=black,
    urlcolor=black,
    pdftitle    = {English Language Profiency Exam},
    pdfsubject  = {Translation of a French paper to English in Numerical Analysis},
    pdfauthor   = {Stephan Grein},
    pdfkeywords = {Translation, English, Frensh, Numerical Analysis, time-parallel methods, Parareal, PDE, ODE}
}

\title{Solution of a PDE by a `parareal' scheme \footnote{Translation of the original French research article \cite{Lions2001} to English by Stephan Grein}}
\author{Jacques-Louis LIONS, Yvon MADAY and Gabriel TURINIC}
\newwatermark*[allpages,color=blue!50,angle=45,scale=2,xpos=0,ypos=0]{DRAFT (Stephan Grein)}

\begin{document}
\maketitle

\begin{abstract}
The purpose of this Note is to propose a time discretization of a partial differential evolution
equation that allows for parallel implementations. The method, based on an Euler scheme,
combines coarse resolutions and independent fine resolutions in time in the same spirit
as standard spacial approximations. The resulting parallel implementation is done in the
non standard time direction. Its main goal concerns real time problems, hence the proposed
terminology of `parareal' algorithm.
\end{abstract}

\tableofcontents
\newpage

\section{Introduction}
\dots
\section{Comments on stability}
\dots
\section{Application of the `parareal' scheme to linear PDEs}
\dots
\section{Analogue application to non-linear PDEs}
\dots

\bibliographystyle{plain}
\bibliography{bib}

%%% content here
\end{document}
